% write you a haskell

\documentclass[11pt, twoside, a4paper]{article}
\author{Hong}
\usepackage{makeidx}
\begin{document}
\index{hello}
Small is beautiful
% \hyphenation{word list}
\newline
\today
\newline 
\TeX
\newline
\LaTeX 
\newline
\LaTeXe
\newline ``Please press the `x' key''
yes---or no? pages 13--54 in-law
\newline 
\newline http://www.rich.edu/\~{}bush 
\newline http://www.clever.edu/$\sim$demo
\newline Not shelfful\\
but shelf\mbox{}ful
Not shelfful
but shelfful\\
\underline{important} \\
FootNotes\footnote{TEST} are often used by people.\\
\emph{text}

% flushleft
% center
\begin{flushright}
This text is right-\\aligned. \LaTeX{} is not trying to make each
line the same length.
\end{flushright}

A typographical rule of thumb
for the line length is:
\begin{quote}
On average, no line should
be longer than 66 characters.
\end{quote}
This is why \LaTeX{} pages have
such large borders by default
and also why multicolumn print
is used in newspapers

The \verb|\ldots| command \ldots
\begin{verbatim}
10 PRINT "HELLO WORLD ";
20 GOTO 10
\end{verbatim}

\begin{verbatim*}
    the starred version of
    the verbatim
    environment emphasizes
    the spaces in the text
\end{verbatim*}



\newpage HHAS
\begin{tabular}{|p{4.7cm}|}
    \hline
    Welcome to Boxy's paragraph.
    We sincerely hope you'll
    all enjoy the show.\\
    \hline
    \end{tabular}
\newline ****************
\newline ****************

    \begin{tabular}{|r|l|}
        \hline
        7C0 & hexadecimal \\
        3700 & octal \\ \cline{2-2}
        11111000000 & binary \\
        \hline \hline
        1984 & decimal \\
        \hline
\end{tabular}

Add $a$ squared and $b$ squared to get $c$ squared. Or, using
a more mathematical approach:
\begin{displaymath}
        c^{2}=a^{2}+b^{2}
\end{displaymath}
or you can type less with:
\[a+b=c\]

\begin{equation} \label{eq:eps}
    \epsilon > 0
\end{equation}

% From (\ref{eq:eps}), we gather
% \ldots{}From \eqref{eq:eps} we
% do the same.

%inline
$\lim_{n \to \infty}
\sum_{k=1}^n \frac{1}{k^2}
= \frac{\pi^2}{6}$

\begin{displaymath}
    \lim_{n \to \infty}
    \sum_{k=1}^n \frac{1}{k^2}
    = \frac{\pi^2}{6}
\end{displaymath}


\begin{equation}
    a^x+y \neq a^{x+y}
\end{equation}
$a\,b\:c\;d\!e$

\begin{displaymath}
    \mathbf{X} =
    \left( \begin{array}{ccc}
    x_{11} & x_{12} & \ldots \\
    x_{21} & x_{22} & \ldots \\
    \vdots & \vdots & \ddots
    \end{array} \right)
\end{displaymath}

\begin{displaymath}
    y = \left\{ \begin{array}{ll}
    a & \textrm{if $d>c$}\\
    b+x & \textrm{in the morning}\\
    l & \textrm{all day long}
    \end{array} \right.
\end{displaymath}

\begin{displaymath}
    \left(\begin{array}{c|c}
    M & D \\
    \hline
    Z & Z
    \end{array}\right)
\end{displaymath}

\begin{eqnarray}
    f(x) & = & \cos x \\
    f'(x) & = & -\sin x \\
    \int_{0}^{x} f(y)dy &
    = & \sin x
\end{eqnarray}

\begin{displaymath}
    {}^{12}_{\phantom{1}6}\textrm{C}
    \qquad \textrm{versus} \qquad
    {}^{12}_{6}\textrm{C}
\end{displaymath}

\begin{displaymath}
    \Gamma_{ij}^{\phantom{ij}k}
    \qquad \textrm{versus} \qquad
    \Gamma_{ij}^{k}
\end{displaymath}

% \theoremstyle{definition} \newtheorem{law}{Law}
% \theoremstyle{plain} \newtheorem{jury}[law]{Jury}
% \theoremstyle{remark} \newtheorem*{marg}{Margaret}
% \begin{law} \label{law:box}
% Don't hide in the witness box
% \end{law}
% \begin{jury}[The Twelve]
% It could be you! So beware and
% see law \ref{law:box}\end{jury}
% \begin{marg}No, No, No\end{marg}

\newpage
\flushleft
\newtheorem{mur}{Murphy}[section]
\begin{mur}
If there are two or more
ways to do something, and
one of those ways can result
in a catastrophe, then
someone will do it.
\end{mur}

% \b

\begin{displaymath}
    \mu, M \qquad \mathbf{P} \qquad
    \mbox{\boldmath $\mu, M$}
\end{displaymath}

% % \begin{displaymath}
% %     \xymatrix{
% %     A \ar[d] \ar[dr] \ar[r] & B \\
% %     D & C }
% \end{displaymath}


\end{document} 

